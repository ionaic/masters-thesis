
%%%%%%%%%%%%%%%%%%%%%%%%%%%%%%%%%%%%%%%%%%%%%%%%%%%%%%%%%%%%%%%%%%% 
%                                                                 %
%                           METHODS                               %
%                                                                 %
%%%%%%%%%%%%%%%%%%%%%%%%%%%%%%%%%%%%%%%%%%%%%%%%%%%%%%%%%%%%%%%%%%% 
 
% \specialhead{METHODS}
\chapter{VISUALIZATION}
\label{chapter:visualization}

% talk about instpiriations here instead of prev work
\section{Inspirations}
\label{section:vis_insp}


\section{Motion Visualization}
\label{section:motion_vis}
For visualizing motion of a character or figure, there are a limited selection of different techniques.  Most common is a sequence of frames in which a character is posed, either in a still sequence or as a video.  As this is a final goal of our system, this is a valid visualization, but fails to provide a simple comparison between one animated sequence and another.  This is desirable for qualifying or quantifying performance of the system.  A sequence of still images is also space-consuming, which can be undesirable for print formats or even digital formats where length or size of document is an issue.

Specific markers can be used to highlight motion of particular parts of the body, such as the pelvis or center of mass.  Other indicators placed on or around the figure can indicate other values, such as arrows to represent vectors of force.  This however can result in clutter within the images, scene, or frame of video, occluding or distraction from the primary animation.
%TODO motion sculptures, see vimeo likes
% the motion sculptures are previous work, but the rest is something I just kind of tried, is that a previous work? should this whole section just be under visualization?

\section{Summary}
\label{section:vis_summary}
