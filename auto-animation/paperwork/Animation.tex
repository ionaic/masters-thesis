%%%%%%%%%%%%%%%%%%%%%%%%%%%%%%%%%%%%%%%%%%%%%%%%%%%%%%%%%%%%%%%%%%% 
%                                                                 %
%                             Animation                           %
%                                                                 %
%%%%%%%%%%%%%%%%%%%%%%%%%%%%%%%%%%%%%%%%%%%%%%%%%%%%%%%%%%%%%%%%%%% 
 
\chapter{ANIMATION}
\label{chapter:animation}

Jumping is the acceleration of a character's center of mass upward.  This motion can be divided into several stages.  First is the lead-up or wind-up stage in which the character flexes or gathers momentum to perform the jump.  This takes the form of a slight crouch (TODO reference cat jumping paper or the background section.  should this be in background?) which prepares the character to exert the necessary force against the ground.  

Next comes the take-off stage.  The character pushes against the floor with their feet, accelerating their center of mass to break contact with the floor.  

\section{Force calculation}
% calculation of the muscle flexion
\subsection{Inverse Kinematic Solving}
As the skeleton is a hierarchy assumed to be rooted at the hip, a problem arises with applying rotations to joints.  To keep a character's feet rooted to the floor as is expected, 


\section{Center of Mass and Balance}

\section{Thrust}