%%%%%%%%%%%%%%%%%%%%%%%%%%%%%%%%%%%%%%%%%%%%%%%%%%%%%%%%%%%%%%%%%%% 
%                                                                 %
%                             Animation                           %
%                                                                 %
%%%%%%%%%%%%%%%%%%%%%%%%%%%%%%%%%%%%%%%%%%%%%%%%%%%%%%%%%%%%%%%%%%% 
 
\chapter{ANIMATION}
\label{chapter:animation}

Jumping is the acceleration of a character's center of mass upward.  This motion can be divided into several stages.  First is the lead-up or wind-up stage in which the character flexes or gathers momentum to perform the jump.  This takes the form of a slight crouch (TODO reference cat jumping paper or the background section.  should this be in background?) which prepares the character to exert the necessary force against the ground.  

Next comes the take-off stage.  The character pushes against the floor with their feet, accelerating their center of mass to break contact with the floor.  

\section{Path Estimation}

\begin{figure}[ht]
	\label{fig:pathEstimate}
	\centering
	\begin{tikzpicture}[node distance = 1em, auto]
% Path Estimate Phase
	%	Path Estimate Stage
    \node [stage] (path) {\nodebox{10em}{Path Estimate \[a = \dfrac{2 (x - x_0 - v_0 t)}{t^2} \]}};
    %	Path Estimate Data
    \node [data, above of=path] (xf) {\nodebox{4em}{Target Position ($x$)}};
    \node [data, left of=xf] (xi) {\nodebox{4em}{Initial Position ($x_0$)}};
    \node [data, left of=xi](t){\nodebox{4em}{Time ($t$)}};
    \node [data, right of=xf] (vi) {\nodebox{4em}{Initial Velocity ($v_0$)}};
    \node [data, below of=path] (accel) {\nodebox{6em}{Target Acceleration ($a$)}};
    
    \path [line] (xi) |- (path);
    \path [line] (xf) -- (path);
    \path [line] (t) |- (path);
    \path [line] (vi) |- (path);
    \path [line] (path) -- (accel);
\end{tikzpicture}
	\caption{Diagram of the path estimation step.}
\end{figure}

\begin{figure}[ht]
	\label{fig:pathExample}
	\caption{Example of a path estimation.}
\end{figure}
Before calculations relating to the model's skeleton are performed, an initial estimate of the jump path is performed.  The estimate uses a simple forward kinematic calculation to determine the force required to move an object through the air from the initial position of the model, denoted as $x_0$ in Figure \ref{fig:pathEstimate}, to a final position, denoted as $x$.  From this force, the acceleration can be determined using $F=ma$ from classical mechanics.

This assumes the character is a rigid body with negligible air resistance acted upon by gravity of $10\frac{m}{s}$.  The 

\section{Windup}
\begin{figure}[ht]
	\label{fig:bendPhase}
	\centering
	\resizebox{\textwidth}{!}{
		\begin{tikzpicture}[node distance = 0.5cm, auto]
% Bend Phase
    % before controller, need to calculate the desired 
    % place nodes
    \node [data, below of=path] (accel) {\nodebox{2.5cm}{Target Acceleration ($a$)}};
	\node [data, left=of accel] (mass) {\nodebox{3.5cm}{Body Mass assigned to each limb ($\left\lbrace m_0, \ldots, m_n\right\rbrace$)}};
    \node [stage, below= of accel] (forceCalc) {\nodebox{6cm}{Calculate desired force $F_{target} = \displaystyle\sum_{j=0}^n {m_j} a$}};
    % connecting lines
    \path [line] (accel) -- (forceCalc);
    \path [line] (mass) |- (forceCalc);
    % ----------------
	
    % PD controller
    % place nodes
    \node [substage, below=4cm of forceCalc] (bendErr) {Calculate error from desired force magnitude ($E_{force}$)};
    \node [substage, left=2.5cm of bendErr] (bendBal) {Calculate balance error ($E_{balance}$)};
    \node [substage, above=of bendBal] (comCalc) {\nodebox{5cm}{Calculate Center of Mass \[C_{mass} = \displaystyle\sum_{\forall m} m \cdot position(m)\]}};
    \node [data, below left=1cm and -2cm of bendErr] (bendErrAll) {\nodebox{4cm}{$E_{all} = E_{force} + E_{balance}$}};
    \node [decision, below=0.5cm of bendErrAll] (bendPDTest) {$E_{all} \overset{?}{\le} \epsilon$};
    \node [substage, left=1cm of bendPDTest] (bendPDEq) {Set new hip position based on $u(i) = k_p E_{all}(i) + k_d \left(E_{all}(i) - E_{all}(i-1)\right)$};
    \node [stage, label={[shift={(-0.5cm, 2cm)}, rotate=90]180:\LARGE PD Controller}, fit=(comCalc) (bendErr) (bendBal) (bendErrAll) (bendPDEq) (bendPDTest)] (bendPD) {};
    \node [data, below=2cm of bendPDTest] (bentSkel) {\nodebox{3cm}{Skeleton in bent position ($\theta_{x} \forall x \in J_m$)}};
    \node [data, below=1cm of bendPDEq] (bendPDConst) {
    \nodebox{4cm}{Proportional and Derivative weights ($k_p, k_d$)}};
    % connecting lines
	\path [line] (forceCalc) -- (bendErr);
	\path [line] (comCalc) -- (bendBal);
	\path [line] (bendErr) |- (bendErrAll);
    \path [line] (bendBal) |- (bendErrAll);
    \path [line] (bendErrAll) -- (bendPDTest);
    \path [line] (bendPDTest) -- (bendPDEq);
    \path [line] (bendPDTest) -- (bentSkel);
    \path [line] (bendPDConst) -- (bendPDEq);
    % ----------------
    
    % CoM inputs
    % place nodes
    \node [data, above =2cm of comCalc] (skeleton) {\nodebox{3cm}{Model with skeleton attached ($J = \left\lbrace j_0, \ldots, j_n \right\rbrace$, contains at least a pelvis and both left and right hips, knees, ankles, heels, and toes)}}; 
    \node [data, left=2cm of skeleton] (mjoints){\nodebox{7cm}{Muscled joints ($J_m = \lbrace$ $j_{pelvis}$, $j_{Lhip}$, $j_{Rhip}$, $j_{Lknee}$, $j_{Rknee}$, $j_{Lankle}$, $j_{Rankle}$, $j_{Lheel}$, $j_{Rheel}$, $j_{Ltoe}$, $j_{Rtoe}$ $\rbrace$}};
    \node [data, left=2cm of mjoints] (muscles) {\nodebox{7cm}{Muscle spring constant for muscled joints ($\left\lbrace k_0, \ldots, k_11\right\rbrace$)}};
	\node [data, left=2cm of muscles] (jconst) {\nodebox{5cm}{Rotation constraints for joints ($\theta_{min}, \theta_{max}$ $\forall j \in J_{muscled}$)}};
	
	% connecting lines
	\path [line] (skeleton) -- (comCalc);
	\path [line] (mjoints) |- (comCalc);
	\path [line] (muscles) |- (comCalc);
	\path [line] (jconst) |- (comCalc);
	% ----------------	
	
	% IK solver side stage
	% place nodes
	\node [data, left=6cmof bendPDEq] (IKCurJoint) {\nodebox{8cm}{$R = position(j)$ $\forall j \in J_{ik} \subseteq J$ starting with the root (hip joint).}};
	\node [data, left=3cm of IKCurJoint] (IKTargetPos) {\nodebox{8cm}{Target position for joint, in this case keeping $E$ in it's original position ($D$).}};
	\node [data, left=3cm of IKTargetPos] (IKEndJoint) {\nodebox{4cm}{Joint to move to target ($E$).}};
	\node [data, above=2cm of IKCurJoint] (IKRD) {\nodebox{6cm}{Normalized vector $\vec{RD}$}};
	\node [data, above=2cm of IKEndJoint] (IKRE) {\nodebox{6cm}{Normalized vector $\vec{RE}$}};
`	\node [substage, above=8cm of IKTargetPos] (IKEq) {$\theta_j = \vec{RD} \times \vec{RE}$.};
	\node [data, above left=3cm of IKEq] (IKItrs) {\nodebox{6cm}{Number of iterations for IK solver ($num_itr$)}};
	\node [data, left=29cm of comCalc] (IKPartBent) {\nodebox{5cm}{Bent skeleton reflecting $u(i)$.}};
	\node [stage, label={[shift={(-1cm, 5.5cm)}, rotate=90]180:\LARGE IK Solver}, fit=(IKCurJoint) (IKEndJoint) (IKTargetPos) (IKItrs) (IKRD) (IKEq) (IKPartBent)] (bendIK) {Requires joints to be in a single chain.};
    % connecting lines
    \path [line] (bendPDEq) -- (IKCurJoint);
    \path [line] (IKPartBent) -- ++(25cm, 0cm) -- ++(0cm, -4.5cm) -| (bendErr);
    \path [line] (IKPartBent) -- (comCalc);
    \path [line] (IKCurJoint) -- (IKRD);
    \path [line] (IKEndJoint) -- (IKRE);
    \path [line] (IKTargetPos) |- (IKRD);
    \path [line] (IKTargetPos) |- (IKRE);
    \path [line] (IKRD) |- (IKEq);
    \path [line] (IKRE) |- (IKEq);
    \path [line] (IKEq) -- (IKPartBent);
    \path [line] (IKItrs) |- (IKEq);
    % ----------------
\end{tikzpicture}
	}
	\caption{Algorithm diagram of the windup phase.}
\end{figure}
\subsection{Center of Mass and Balance}
\subsection{Force calculation}
% calculation of the muscle flexion


\subsection{Inverse Kinematic Solving}
As the skeleton is a hierarchy assumed to be rooted at the hip, a problem arises with applying rotations to joints.  To keep a character's feet rooted to the floor as is expected, 


\section{Thrust and Takeoff}