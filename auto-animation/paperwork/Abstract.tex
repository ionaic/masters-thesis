%%%%%%%%%%%%%%%%%%%%%%%%%%%%%%%%%%%%%%%%%%%%%%%%%%%%%%%%%%%%%%%%%%%
%                                                                 %
%                            ABSTRACT                             %
%                                                                 %
%%%%%%%%%%%%%%%%%%%%%%%%%%%%%%%%%%%%%%%%%%%%%%%%%%%%%%%%%%%%%%%%%%%

\specialhead{ABSTRACT}
Animating a character for video games and films is difficult and time consuming, requiring hours of artist labor to produce each animation.  These animations are set and inflexible, requiring changes to the animation or sometimes fully new animations to suit new characters or situations for natural looking movements.  Jumping is one such animation, where the size, mass, strength, and environment affect the movement of the character.  Traditionally these animations are produced by manually posing the character for certain key frames and interpolating between the frames to produce a smooth animation.  The more detailed or lengthy an animation, the more work required to specify it.  Physics-based simulation for animation production can reduce this work, creating animations for a variety of situations based on constants set for the character and environment.  These animations can then be easily recreated or adjusted for different environments by changing the constants set for generation.

This thesis work presents a simulation-based method of control for a character, focusing on the lower body, to produce jumping animations for a variety of situations and body parameters.  Two methods of simulation are described, one using a torque calculation and the other using an energy calculation to determine poses for the character.  Our simulation takes as input a mesh representing the character, a tree of joints describing the skeleton, a set of muscles, mass assigned to each limb of the body, and a description of the desired path through desired timings, gravity, and desired displacement.  An inverse kinematic solver is used to aid in posing the character.

Contributions of this thesis include an implementation of a simulation to produce jump animations in \unity{}, a description of character poses based on torque as well as another based on energy, a sampling-based method for choosing a target position, and visualization of the produced animations in several ways to aid in debugging, analysis and presentation.

%When dealing with emergency situations in the wake of a disaster where the infrastructure of a region is damaged, a wide range of expertise is needed to quickly and efficiently solve issues. A prototype system for training emergency response personnel was designed to foster a collaborative environment through a single visualization with multiple users. The traditional methods of providing more detail in geographic data when dealing with a single user do not work in situations requiring interactions from multiple users, as this requires the loss of detail elsewhere within the fixed resolution of a display. The data being displayed in this simulation consists of a graph network of nodes and edges on top of a underlying series of satellite images. The nodes and edges are drawn over corresponding physical locations on the satellite images. When interacting with this system, it is often helpful to zoom in and increase the level of detail for two primary reasons: it is difficult to distinguish geographic features at low resolution, and subsequently, it is difficult to distinguish between nodes when the screen distance between them is relatively small. 

%This thesis work presents a method for providing a focus plus context solution to this system, allowing for continuous visual information with a minimal amount of distortion. This method creates circular areas of high magnification which gradually fall off to a base level surrounding individual cursors. These different regions achieve the original goals of zooming by providing a magnified look at the satellite images and increasing the screen distance between nodes. By having these regions centered on cursors, multiple users can view data on a shared display with their own degree of magnification while still retaining the ability to view the majority of the surrounding data.

%Contributions of this thesis include efficient implementations of rendering a graph network and text onto an underlying layer of satellite images, algorithms to perform the transformations of edges, vertices, and image data for rendering, and a preliminary feedback on the usability of these changes along with suggestions for a formal user study to be conducted as future work.
