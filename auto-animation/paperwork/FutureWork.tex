
%%%%%%%%%%%%%%%%%%%%%%%%%%%%%%%%%%%%%%%%%%%%%%%%%%%%%%%%%%%%%%%%%%% 
%                                                                 %
%                           FUTUREWORK                            %
%                                                                 %
%%%%%%%%%%%%%%%%%%%%%%%%%%%%%%%%%%%%%%%%%%%%%%%%%%%%%%%%%%%%%%%%%%% 
 
% \specialhead{FUTUREWORK}
\chapter{FUTURE WORK AND CONCLUSION}
\label{chapter:future_work}
In this chapter we discuss future work and conclude our discussion of research on this simulation.  Section \ref{section:future_work} discusses possible and planned work on this topic, as well as potential future work in this and closely related lines of study.  Section \ref{section:conclusion} concludes this document with final thoughts and a summary of the document.

\section{Future Work}
\label{section:future_work}
\begin{itemize}
	\item learning model - slight adjustments to different situations can be more easily generated through the use of a learning model such as in XYZ \cite{falling_landing, muscle_based_bipeds}.
	\item upper body affects, non-negligible effect of upper body movements on acceleration of CoM
	\item expand to non-bipedal
	\item more complex muscle and motion model, trials to determine level of effect on simulation (is it worthwhile)
	\item in-air phase: add complexity, can we jump to create vaulting motions, mid-jump push off from objects (jump and push off a wall), rotations, or flips.
\end{itemize}

We planned some further work that we decided was beyond the scope of this thesis.  Our current system does not allow much flexibility with specification of the path the character travels for its jump.  Jumping path estimation could be performed based on a policy.  Possible policies are achieving a height while jumping to a target destination, pathing to clear an object or intersect with an object, follow a path defined by the user, and jumping with a user specified velocity or speed.  These policies would require a smarter handling of the in air phase of the jump, which would be best implemented as a secondary controller to allow controller composition for more complex motions.  A more complex in air controller would ideally handle cases such as acrobatics, in air maneuvers, and checking for and handling collisions.

Current work exists for landing motions such as \liufall

\section{Conclusion and Summary}
\label{section:conclusion}