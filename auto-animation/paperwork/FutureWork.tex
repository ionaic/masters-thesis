
%%%%%%%%%%%%%%%%%%%%%%%%%%%%%%%%%%%%%%%%%%%%%%%%%%%%%%%%%%%%%%%%%%% 
%                                                                 %
%                           FUTUREWORK                            %
%                                                                 %
%%%%%%%%%%%%%%%%%%%%%%%%%%%%%%%%%%%%%%%%%%%%%%%%%%%%%%%%%%%%%%%%%%% 
 
% \specialhead{FUTUREWORK}
\chapter{FUTURE WORK AND CONCLUSION}
\label{chapter:future_work}

\begin{itemize}
	\item learning model - slight adjustments to different situations can be more easily generated through the use of a learning model such as in XYZ \cite{falling_landing, muscle_based_bipeds}.
	\item upper body affects, non-negligible effect of upper body movements on acceleration of CoM
	\item expand to non-bipedal
	\item more complex muscle and motion model, trials to determine level of effect on simulation (is it worthwhile)
	\item in-air phase: add complexity, can we jump to create vaulting motions, mid-jump push off from objects (jump and push off a wall), rotations, or flips.
\end{itemize}


\section{Conclusion}
\label{section:conclusion}