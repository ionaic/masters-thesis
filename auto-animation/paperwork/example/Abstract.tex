%%%%%%%%%%%%%%%%%%%%%%%%%%%%%%%%%%%%%%%%%%%%%%%%%%%%%%%%%%%%%%%%%%%
%                                                                 %
%                            ABSTRACT                             %
%                                                                 %
%%%%%%%%%%%%%%%%%%%%%%%%%%%%%%%%%%%%%%%%%%%%%%%%%%%%%%%%%%%%%%%%%%%

\specialhead{ABSTRACT}
When dealing with emergency situations in the wake of a disaster where the infrastructure of a region is damaged, a wide range of expertise is needed to quickly and efficiently solve issues. A prototype system for training emergency response personnel was designed to foster a collaborative environment through a single visualization with multiple users. The traditional methods of providing more detail in geographic data when dealing with a single user do not work in situations
requiring interactions from multiple users, as this requires the loss of detail elsewhere within the fixed resolution of a display. The data being displayed in this simulation consists of a graph network of nodes and edges on top of a underlying series of satellite images. The nodes and edges are drawn over corresponding physical locations on the satellite images. When interacting with this system, it is often helpful to
zoom in and increase the level of detail for two primary reasons: it is difficult to distinguish geographic features at low resolution,
and subsequently, it is difficult to distinguish between nodes when the screen distance between them is relatively small. 

This thesis work presents a method for providing a focus plus context
solution to this system, allowing for continuous visual information with a minimal amount of distortion. This method creates circular
areas of high magnification which gradually fall off to a base level surrounding individual cursors.
These different regions achieve the original goals of zooming by providing a magnified look at
the satellite images and increasing the screen distance between nodes. By having these regions
centered on cursors, multiple users can view data on a shared display with their own degree
of magnification while still retaining the ability to view the majority of the surrounding data.

Contributions of this thesis include efficient implementations of rendering a graph network 
and text onto an underlying layer of satellite images, algorithms to perform the transformations of edges, vertices, and image data for rendering, and a preliminary feedback on the usability of these changes along with suggestions for a formal user study to be conducted as future work.
