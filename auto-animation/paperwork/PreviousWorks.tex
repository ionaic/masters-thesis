
%%%%%%%%%%%%%%%%%%%%%%%%%%%%%%%%%%%%%%%%%%%%%%%%%%%%%%%%%%%%%%%%%%% 
%                                                                 %
%                           PREVIOUS WORK                         %
%                                                                 %
%%%%%%%%%%%%%%%%%%%%%%%%%%%%%%%%%%%%%%%%%%%%%%%%%%%%%%%%%%%%%%%%%%% 
 
% \specialhead{PREVIOUS WORK}
\chapter{PREVIOUS WORK}
\label{chapter:previous_work}
In this chapter, I introduce previous work in the field that inspired and directed my work. First I discuss the production of animation using motion capture in Section \ref{subsection:mocap_bg}, then muscle-based simulations in Section \ref{subsection:muscle_sim_bg}.  Muscle-based simulations were a major inspiration for my work, as well as rigid-body simulations that control the character through means other than muscle simulations which I describe in Section \ref{subsection:rigid_body_bg}.

\section{Background of Computer Generated Animation}
\label{section:computer_gen_bg}
\subsection{Motion Capture}
\label{subsection:mocap_bg}
Producing athletic animations for human characters is difficult.  Motion capture is one method used for production of realistic animations for human athletics and other motions. Cameras are used to capture the movements of live actors that are then applied to 3D models, allowing the actor's performance to control the virtual character.  In marker-based capture, numerous tracking points are attached to the actor, aiding a system of connected cameras in following the movements of the actor.  Other methods attempt pose estimation without markers by extracting silhouettes and edges of the actor from images, though these methods are less capable of capturing detail. Richard Radke gives an overview of these methods, as well as of the necessary calibration of the systems and processing of the data collected \cite{radke}.

In marker-based capture optical marks are worn by the actor.  These marks are then used to determine the position of the actor in 3-dimensional space by a set of cameras surrounding the actor.  The camera system must be calibrated before capture can be performed.  The markers are then tracked by the camera system, estimating positions in space by leveraging multiple cameras' views of a each marker.  One method of tracking these markers is framed as a learning problem as per Liu and McMillan, using principal component analysis to build a linear model \cite{liu_mocap}.  

After acquiring the marker data, the pose of the body is estimated using forward kinematics.  Inverse kinematics is then used to determine the angles of the skeleton's joints from the tracked positions of the skeleton.  I use inverse kinematics as part of my simulation, and so discuss relevant background in Section \ref{subsection:ik_bg}.  However, the problem for motion capture is slightly different as the skeleton is usually over-constrained by many markers, as opposed to the inverse kinematics problem in my simulation where the problem is to position a handful of points on the skeleton at specified target positions.  The same principles apply in these situations, but utilize cost functions to minimize error across the entire set of markers and learning models to determine the most likely motion the actor is performing.

The motion captured must then be applied to the virtual model.  Several separately recorded motions may need to be blended together to produce longer animation sequences, requiring interpolation and blending between captured motions.  Witkin and Popovi\'{c} provide one method for blending and editing both keyframe and captured animations using motion curves, which are descriptions of the parameters of the model over time.

In marker-less capture, the images are segmented to determine the position of the actor.  Depth sensing may be used, such as that described by Shotton et al \cite{shotton_kinect}, which is used in the Kinect created by Microsoft.  This method utilizes a large amount of training data, both captured from humans and synthesized, to build a decision forest to be used for classification.

Motion capture overall provides a strong method of creating realistic animations for characters, but has several shortcomings.  For each animation desired, a capture must be made, requiring large amount of actor or performer time.  Additionally, the equipment for motion capture systems can be costly both in money and time, requiring large amounts of setup and calibration for a capture session.  These motions also require positioning of markers for detailed capture, necessitating props and prostheses in cases where the actor is not proportioned like the character desired, e.g. when an actor plays another species with longer limbs than a human.

\subsection{Muscle-based Simulation}
\label{subsection:muscle_sim_bg}
Another method of producing animations simulates a complex muscle model mimicking the biology of the human body. Muscle-based approaches produce realistic motions which adapt to the environment, using a complex model of the musculo-skeletal structure. These methods are often complicated to produce, requiring learning methods and complex constructions of the character's body, though they react well to applied stimuli and are often flexible in the animations produced.

Grzeszczuk and Terzopoulos produce animations of animals, mostly those with many degrees of freedom such as fish, by producing actuator control functions for the muscles of the controlled character \cite{fish_muscles}.  An objective function provides feedback which can then be used to learn muscle activations, modeling neural signals, required to produce motions.  Controllers are then learned for low-level tasks such as moving at different speeds and turning.  By composing numerous learned controllers, complex motions are produced such as a fish jumping out of the water.

Geijtenbeek et al.\cite{muscle_based_bipeds} use a rough, user-created muscle routing on a skeleton to produce various gaits that are learned based on the velocity and environment.  The muscle routing is optimized to remain within a physical region while providing optimal forces on the skeleton based on freedom of motion of the skeletal joints and the calculated optimal length of the muscle.  This model is then used to compute sequences of muscle activations that produce the final animations.  This method is effective, producing good results in various levels of gravity on at least 10 different bipedal skeletons which can react to external stimuli.

My simulation utilizes a muscle-based approach, though in a simpler manner than in the work described above.  I use a simplified muscle model to reduce user-specified biology and to reduce complexity of the problem of moving the character. By avoiding a learning method, I reduced time spent on precomputation and also attempted to reduce overall compute time to achieve interactive performance.  I chose the simpler model also to limit the scope of this thesis, though attempting similar work with a learning model and complex muscle model would provide interesting work in the future.

%\begin{figure}[htp]
	%\centering
	%\includegraphics[width=0.2\columnwidth]{muscle_based/muscle_routing.eps}
	%\caption{Example of a muscle routing on a skeleton from Geijtenbeek et al. \cite{muscle_based_bipeds}.}
%\end{figure}
		
%\begin{figure}[htp]
	%\centering
	%\includegraphics[width=0.3\columnwidth]{falling_motion/falling1.eps}
	%\hspace{0.1\columnwidth}
	%\includegraphics[width=0.3\columnwidth]{falling_motion/falling3.eps}
	%\caption{Breakdown of a hands-first falling approach from Ha et al. \cite{falling_landing} and of a feet-first landing approach.  Ha et al. use a rolling strategy to minimize stress on the body and produce a realistic fall.}
%\end{figure}


\subsection{Non-Muscle Simulations}
\label{subsection:rigid_body_bg}
Instead of complex muscle systems, some physical simulations utilize a rigid-body character with a user-defined skeleton to find optimal poses based on desired conditions other than muscle simulation.  \liufall{} utilize such a scheme to generate landing motions for human characters based off linear velocity, global angular velocity, and angle of attack  \cite{falling_landing}.  The system chooses either a feet first or hands first landing strategy and moves into a roll to reduce stress on the body using principles from biomechanics and robotics.  A sampling method is applied to determine successful conditions, producing bounding planes for the data.  The movement is broken into stages of airborne and landing, in which the character re-positions for the designated landing strategy, and executes the landing strategy respectively. Each of these is separated into impact, roll, and get-up stages.  Movement and joint positions are produced using PID servos.  

Other work on producing such controllers was produced by Faloutsos et al. who described a method of composing such controllers by giving pre-conditions, post-conditions, and intermediate state requirements \cite{composable_controllers}.  The composed controllers are then chosen at each step based on the current pose and which controller is deemed most suitable.  By providing a state-machine-style construction for the controllers, they create a way to build many smaller controllers into a more complex motion.

Hodgins et al. created several controllers for running, vaulting, and bicycling, creating realistic motions and secondary motion using rigid bodies and spring-mass simulations \cite{anim_human_athletics}.  Geijtenbeek and Pronost provide a detailed review of physics based simulations \cite{inter_physics_anim}.

Koga et. al use path planning, inverse kinematics, and forward simulation to generate animations of arm motions for robots and humans working cooperatively.  They produce arm manipulations that avoid collisions and result in final positions and orientations for specified parts of the arm to produce motions such as a human putting on glasses and a robot arm and human cooperating to flip a chessboard \cite{motion_intentions}.

My work was inspired heavily by these simulations, though I chose to incorporate muscles into the simulation.  These control schemes helped inspire the method of control and optimization used in my simulation.  The descriptions of controllers as modules that can be fit together into a larger state machine of controllers to handle many situations led to the creation of my controller as one such module, intended to be used alongside other controllers for handling landing and in-air maneuvers such as that created by \liufall{}.

\subsection{Inverse Kinematics}
\label{subsection:ik_bg}
Solving the inverse kinematics problem is necessary to my simulation for positioning of the feet.  As described in Section \ref{section:ik} in more detail, the setup of my skeleton requires a method of posing the character such that if the pelvis is moved to a different position, the character's feet remain in the same place.  Several different methods exist of varying complexity.  A simple, cyclic-coordinate-descent method allows solving for a single chain of joints as described by Lander \cite{kine1, kine2}. A detailed description of this method can be found in \ref{section:ik}.  This method was chosen for simplicity and the minimal extra infrastructure required, as other methods required matrix math libraries not easily available.

Buss surveys inverse kinematics methods, describing the classical Jacobian transpose, Jacobian pseudoinverse, and damped least squares methods \cite{buss_ik}.  End effectors are defined as particular points on the skeleton.  The positions of the end effectors are defined by the joint angles of the skeleton.  The Jacobian matrix is a function of the angles of the joints, describing the relationship between the joint angles and the position of each end effector.  The Jacobian for a particular state of the skeleton can be computed, with further Jacobian matrices computed by choosing changes in the angles of the skeleton.  The inverse of the Jacobian gives a value for this choice of change.  As the Jacobian is likely not invertible, approximations and alternatives are used.  The transpose provides a fast approximation, though it is a different solution than the inverse.  The pseudoinverse provides a good, fast approximation, but lacks stability around singularities.  The damped least squares method avoids this issue of stability by incorporating a damping constant.

Another method by Aristidou and Lasenby frames the problem as finding a point on a line \cite{fabrik}.  The system, termed Forward And Backward Reaching Inverse Kinematics (FABRIK) uses an iterative approach.  Iterating over the the joints of the body, the algorithm repeatedly finds a new joint position along the line between the current and desired positions with an appropriate distance from its neighbor.  Which neighbor is used for calculation is determined by if the iteration is currently working forwards or backwards.

\section{Commercial Software}
\label{section:commercial_bg}
Several technologies exist to similarly aid in animation production. \unity{} Mecanim applies constructed animations of various types to similar skeletons, providing joint constraints and muscle definitions in a similar manner to my simulation \cite{unity_mecanim}.  MecAnim provides functionality for constraining range of motion and blending between existing animations, utilizing existing clips to produce complex animations in a manner similar to the composable controllers described by Faloutsos et al., as well as inverse kinematics solving.  Due to a lack of understanding of the features and limitations, as well as what level of control is available, I chose not to utilize Mecanim for my current implementation.  As discussed in chapter \ref{chapter:future_work}, future work would ideally take advantage of Mecanim.

3ds Max footsteps offer a method of positioning feet and producing walk, run, and jump cycles based on  number of parameters \cite{3dsmax}.  Without knowledge of the algorithm, analysis is difficult, but it seems to produce animations by specifying timing and parameters about the stride.  Parameters defined include stride width, length, and height.  My simulation seeks to produce more natural looking animations through use of a muscle simulation.  The footstep style of animation may be preferable to artists however as it gives very strong control over the timing and spacing of the individual events of the animation, such as foot falls.

\section{Summary}
\label{section:background_summary}
In this chapter I discussed the previous works that guided my research as well as existing solutions to the problem of animating characters. I discussed existing work in motion capture, which offers a method of producing realistic animations, but must be performed offline and requires significant actor time as well as setup of a capture system.  I described muscle and non-muscle simulations used to generate animations, both of which inspired the work described in Chapter \ref{chapter:animation}, and discussed approximate solutions to the inverse kinematics problem, which is expanded upon in Section \ref{section:ik}.  Finally, I discussed some existing solutions and tools in commercial software which perform similar tasks to the work in this thesis, but ultimately fill different roles in the production of animation.  In the next chapter, I discuss my simulation in detail.