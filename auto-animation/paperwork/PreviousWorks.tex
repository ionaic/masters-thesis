
%%%%%%%%%%%%%%%%%%%%%%%%%%%%%%%%%%%%%%%%%%%%%%%%%%%%%%%%%%%%%%%%%%% 
%                                                                 %
%                           PREVIOUS WORK                         %
%                                                                 %
%%%%%%%%%%%%%%%%%%%%%%%%%%%%%%%%%%%%%%%%%%%%%%%%%%%%%%%%%%%%%%%%%%% 
 
% \specialhead{PREVIOUS WORK}
\chapter{PREVIOUS WORK}
\label{chapter:previous_work}

\section{Background}
Producing athletic animations for human characters is difficult.  One method, motion capture is used for production of realistic animations for human athletics and other motions, however it requires the collection of information for each motion and does not adapt to the virtual environment.  Muscle-based approaches produce realistic motions which adapt to the environment, using a complex model of the musculo-skeletal structure.  Geijtenbeek et al. use a rough, user created muscle routing on a skeleton to produce various gaits that are learned based on the velocity and environment.  The muscle routing is optimized to remain within a region while providing optimal forces on the skeleton based on freedom of motion of the skeletal joints and the calculated optimal length of the muscle.  This model is then used to compute sequences of muscle activations, modeling neural signals, which produce the final animations.  This method is effective, producing good results in various levels of gravity on at least 10 different bipedal skeletons \cite{muscle_based_bipeds}.

%\begin{figure}[htp]
	%\centering
	%\includegraphics[width=0.2\columnwidth]{muscle_based/muscle_routing.eps}
	%\caption{Example of a muscle routing on a skeleton from Geijtenbeek et al. \cite{muscle_based_bipeds}.}
%\end{figure}
		
Inverse kinematics approaches attempt to generate the motion based on a desired final position, determining the skeletal position by solving the system given constraints.  Koga et. al use path planning, inverse kinematics, and forward simulation to generate animations of arm motions for robots and humans working cooperatively.  They produce arm manipulations that avoid collisions and result in final positions and orientations for specified parts of the arm to produce motions such as a human putting on glasses and a robot arm and human cooperating to flip a chessboard \cite{motion_intentions}.  
		
%\begin{figure}[htp]
	%\centering
	%\includegraphics[width=0.3\columnwidth]{falling_motion/falling1.eps}
	%\hspace{0.1\columnwidth}
	%\includegraphics[width=0.3\columnwidth]{falling_motion/falling3.eps}
	%\caption{Breakdown of a hands-first falling approach from Ha et al. \cite{falling_landing} and of a feet-first landing approach.  Ha et al. use a rolling strategy to minimize stress on the body and produce a realistic fall.}
%\end{figure}
		
Physical simulations utilize a rigid-body character with a user-defined skeleton to find optimal poses based on desired conditions.  Ha et al. utilize such a scheme to generate landing motions for human characters based off linear velocity, global angular velocity, and angle of attack.  The system chooses either a feet first or hands first landing strategy and moves into a roll to reduce stress on the body using principles from biomechanics and robotics.  A sampling method is applied to determine successful conditions, producing bounding planes for the data.  The movement is broken into stages of airborne and landing, in which the character re-positions for the designated landing strategy, and executes the landing strategy respectively. Each of these is separated into impact, roll, and get-up stages.  Movement and joint positions are produced using PID servos \cite{falling_landing}.  Other work on producing such controllers was produced by Faloutsos et al. who described a method of composing such controllers by giving pre-conditions, post-conditions, and intermediate state requirements.  The composed controllers are then chosen at each step based on the current pose and which controller is deemed most suitable \cite{composable_controllers}.  Hodgins et al. created several controllers for running, vaulting, and bicycling, creating realistic motions and secondary motion using rigid bodies and spring-mass simulations \cite{anim_human_athletics}.  Geijtenbeek and Pronost provide a detailed review of physics based simulations \cite{inter_physics_anim}.

\subsection{Existing Technologies}
Unity3D mecanim applies constructed animations of various types to similar skeletons.  This requires your animation to be constructed manually or generated through a system before it can be applied.

3ds max footsteps offer a method of positioning feet and producing rudimentary animations, but require manual input and tweaking for a a full animation.